\addcontentsline{toc}{chapter}{Formelzeichen}
%\vspace*{0.025cm}
\chapter*{Formelzeichen}
\section*{Allgemeine Hinweise}
%
%\begin{spacing}{1.5}
%	\setlength\LTleft{100pt}
%	\setlength\LTright{\fill}
%	%\begin{longtable}{@{\extracolsep{\fill}}cl}
%	\begin{longtable}{cl}
%		$\dot{X}$				&zeitliche Ableitung\\
%		$\hat{X}$				&Amplitudenwert \\
%		$\vec{X}$				&Vektor \\
%		$\underline{X}$			&Komplexer Phasor\\
%		$\underline{X}^*$ 		&Konjugiert-Komplexer Phasor\\
%		$\textbf{X}$			&Matrix\\
%	\end{longtable}
%\end{spacing}
%%
%\section*{Indizes}
%%
%Treten mehrere der hier aufgelisteten Indizes auf, so werden diese mit Komma getrennt angeschrieben.
%\begin{spacing}{1.5}
%	\setlength\LTleft{100pt}
%	\setlength\LTright{\fill}
%	%\begin{longtable}{@{\extracolsep{\fill}}cl}
%	\begin{longtable}{cl}
%		\textsf{A}				&Pol A\\
%		\textsf{B}				&Pol B\\
%		\textsf{B}				&Gleichungssystem magn. Flussdichte (nur in Kapitel \ref{ch_04MagPotRot})\\
%		\textsf{d}				&$d$-Achse/Komponente\\
%		\textsf{Fe}				&Eisen(-blech)\\
%		\textsf{fb}				&Flusssperre (flux-barrier)\\
%		\textsf{fbe}			&Flusssperrenende (flux-barrier end)\\
%		\textsf{fc}				&Flussleitpfad (flux-carrier)\\
%		\textsf{H}				&Gleichungssystem magn. Feldstärke (nur in Kapitel \ref{ch_04MagPotRot})\\
%		$i$						&Nummer der Flusssperre gezählt von außen nach innen: 1,2,...,$N_\textsf{fb}$\\
%		$j$						&Nummer des Abschnitts für Unterteilung des Flusssperrenendes: 1,2,...\\
%		\textsf{q}				&$q$-Achse/Komponente\\
%		\textsf{R}				&Rotor\\
%		\textsf{S}				&Stator\\
%		\textsf{U}				&Strang U\\
%		\textsf{V}				&Strang V\\
%		\textsf{W}				&Strang W\\
%		\textsf{\delta}			&Luftspalt\\
%		\textsf{\nu}			&Ordnungszahlen der Grund- und Oberwellen im Luftpalt\\
%		\textsf{\xi}			&Ordnungszahlen der Grund- und Oberwellen im Luftpalt\\
%	\end{longtable}
%\end{spacing}
%%
%%
%\section*{Lateinische Formelzeichen}
%\begin{spacing}{1.5}
%	\setlength\LTleft{8pt}
%	\setlength\LTright{\fill}
%	\begin{longtable}{@{\extracolsep{\fill}}ccl}
%		\textbf{Symbol}		&\textbf{Einheit}			&\textbf{Beschreibung}                                           \\\midrule\endhead
%		$A$ 				&$\textsf{m}^{2}$	&Fläche\\
%		$A_{j}$ 		&$\textsf{m}^{2}$	&Querschnittsfläche an der $j$-ten Stützstelle des Flusssperrenendes\\
%		$A_\textsf{S}$ 		&$\frac{\textsf{A}}{\textsf{m}}$	&Strombelag\\
%		$\tilde{A}_\textsf{S}$&$\frac{\textsf{A}}{\textsf{m}}$	&äquivalenter Strombelag\\
%		$A_\textsf{z}$ 		&$\frac{\textsf{V\cdot s}}{\textsf{m}}$	&Komponente des magn. Vektorpotentials normal zur Blechebene \\
%		$a$ 				&$\textsf{-}$	&Anzahl paralleler Zweige\\
%		$a$ 				&$\frac{\textsf{m}}{\textsf{s}^\textsf{2}}$	&Beschleunigung\\
%		$a_\eta$, $a_\zeta$ &$\frac{\textsf{m}}{\textsf{s}^\textsf{2}}$	&Beschleunigung in Richtung $\eta$, in Richtung $\zeta$\\
%		$B$ 				&$\textsf{T}$	&magnetische Flussdichte\\
%		$B_{j\textsf{a}}$, $B_{j\textsf{b}}$ 		&$\textsf{T}$	&magn. Flussdichte an der $j$-ten Stützstelle $\textsf{a}$,$\textsf{b}$ des Flusssperrenendes\\
%		$B_\textsf{n}$ 		&$\textsf{T}$	&magnetische Flussdichte in Normalenrichtung\\
%		$B_\textsf{t}$ 		&$\textsf{T}$	&magnetische Flussdichte in Tangentenrichtung\\
%		$B_\delta$ 			&$\textsf{T}$	&magnetische Luftspaltflussdichte\\
%		$b_\textsf{Q}$ 		&$\textsf{m}$	&Nutschlitzbreite\\
%		$b_\textsf{0}$ 		&$\textsf{rad}$	&äquivalente Nutschlitzbreite im Bogenmaß (mech.)\\
%		$d_\textsf{A}$ 		&$\textsf{m}$	&Außendurchmesser\\
%		$d_\textsf{Sh}$ 	&$\textsf{m}$	&Wellendurchmesser\\
%		$d_\textsf{Si}$ 	&$\textsf{m}$	&Statorbohrungsdurchmesser\\
%		$E$ 				&$\frac{\textsf{N}}{\textsf{m}^\textsf{2}}$		&Elastizitätsmodul\\
%		$F_{\textsf{L}\eta}$, $F_{\textsf{L}\zeta}$&$\textsf{N}$&Kraft auf das linke Lager in Richtung $\eta$, in Richtung $\zeta$\\
%		$F_{\textsf{R}\eta}$, $F_{\textsf{R}\zeta}$&$\textsf{N}$&Kraft auf das rechte Lager in Richtung $\eta$, in Richtung $\zeta$\\
%		$f_\textsf{t}$ 		&$\frac{\textsf{N}}{\textsf{m}^\textsf{2}}$	&\textit{Maxwell}'sche Schubspannung\\
%		$H$ 				&$\frac{\textsf{A}}{\textsf{m}}$	&magnetische Feldstärke\\
%		$H_{j\textsf{a}}$, $H_{j\textsf{b}}$ 		&$\frac{\textsf{A}}{\textsf{m}}$	&magn. Feldstärke an der $j$-ten Stützstelle $\textsf{a}$,$\textsf{b}$ des Flusssperrenendes\\
%		$H_\textsf{t}$ 		&$\frac{\textsf{A}}{\textsf{m}}$	&magnetische Feldstärke in Tangentenrichtung\\
%		$H_\delta$ 			&$\frac{\textsf{A}}{\textsf{m}}$	&magnetische Feldstärke im Luftspalt\\
%		$I_\textsf{S}$		&$\textsf{A}$	&Strangstrom Effektivwert\\
%		$i_\textsf{S}$ 		&$\textsf{A}$	&Strangstrom Augenblickswert\\
%		$i_\textsf{C}$ 		&$\textsf{A}$	&Spulenstrom Augenblickswert\\
%		$j$ 				&$\textsf{-}$	&imaginäre Einheit\\
%		$J_{\textsf{H,fc,}i}$, $J_{\textsf{M,fc,}i}$, $J_{\textsf{L,fc,}i}$&$\textsf{-}$	&Integrale für Berechnung der magn. Potentiale der Flussleitpfade\\
%		$K_\delta$			&$\frac{\textsf{V\cdot s}}{\textsf{A}}$				&Konstante für Berechnung der magn. Potentiale der Flussleitpfade\\
%		$k_\textsf{C}$ 			&$\textsf{-}$	&\textit{Carter}-Faktor\\
%		$k_{\textsf{d,}\nu}$&$\textsf{-}$	&Zonenfaktor\\
%		$K_{h\textsf{,}i}$	&$\textsf{-}$	&Proportionalitätsfaktor der $h$-ten Drehmomentharmonischen\\
%		 & &für die $i$-te Flusssperre bei Rotorvariante 1\\
%		$K_{h\textsf{,}i\textsf{,AB}}$	&$\textsf{-}$	&Proportionalitätsfaktor der $h$-ten Drehmomentharmonischen\\
%		 & &für die $i$-te Flusssperre in Pol \textsf{A} und \textsf{B} bei Rotorvariante 2\\
%		$k_{\textsf{p,}\nu}$&$\textsf{-}$	&Sehnungsfaktor\\
%		$k_\textsf{Q}$		&$\textsf{-}$	&Verhältnis Statornutbreite zu Nutteilung\\
%		$k_{\textsf{w,}\nu}$&$\textsf{-}$	&Wicklungsfaktor\\
%		$k_\textsf{w,fc}$	&$\textsf{-}$	&Skalierungsfaktor Flussleitpfadbreite an der $q$-Achse\\
%		$k_\textsf{w,q}$	&$\textsf{-}$	&Verhältnis von Eisen zu Luft in der $q$-Achse\\
%		$k_\textsf{Y}$		&$\textsf{-}$	&Verhältnis Nuttiefe zu Differenz Statoraußen- zu Statorinnenradius\\
%		$L$					&$\textsf{H}$	&Selbsinduktivität\\
%		$l_{\textsf{fb,}i}$ 	&$\textsf{m}$	&Länge der $i$-ten Flusssperre\\
%		$l_\textsf{Fe}$ 	&$\textsf{m}$	&Aktivlänge\\
%		$l_\textsf{Q}$ 		&$\textsf{m}$	&Statornuttiefe\\
%		$l_\textsf{R}$ 		&$\textsf{m}$	&Länge des Rotors\\
%		$M_\textsf{e}$		&$\textsf{Nm}$	&elektromagnetisches Drehmoment\\
%		$M_\textsf{e,max}$	&$\textsf{Nm}$	&maximales Drehmoment\\
%		$M_\textsf{e,mean}$	&$\textsf{Nm}$	&Mittelwert des elektromagnetischen Drehmoments\\
%		$M_{h}$		&$\textsf{Nm}$	&elektromagnetisches Drehmoment der $h$-ten Harmonischen\\
%		$m$					&$\textsf{-}$	&Strangzahl\\
%		$m$					&$\textsf{kg}$	&Masse\\
%		$m_\textsf{R}$		&$\textsf{kg}$	&Rotormasse\\
%		$N$ 				&$\textsf{-}$	&magnetisch wirksame Strangwindungszahl\\
%		$N_\textsf{fb}$ 	&$\textsf{-}$	&Anzahl Flusssperren pro Pol\\
%		$N_\textsf{C}$ 		&$\textsf{-}$	&Spulenwindungszahl\\
%		$P_\textsf{e}$ 		&$\textsf{W}$	&elektrische Wirkleistung\\
%		$P_\textsf{m}$ 		&$\textsf{W}$	&mechanische Leistung\\
%		$p$					&$\textsf{-}$	&Polpaarzahl\\
%		$p_\textsf{F,A}$ 	&$\frac{\textsf{N}}{\textsf{m}^\textsf{2}}$		&Fugendruck\\
%		$p_\textsf{F,PA}$ 	&$\frac{\textsf{N}}{\textsf{m}^\textsf{2}}$		&Fugendruck bei vollständiger Plastifizierung des Außenteils\\
%		$p_\textsf{min}$ 	&$\frac{\textsf{N}}{\textsf{m}^\textsf{2}}$		&Mindestfugendruck\\
%		$Q$ 				&$\textsf{-}$	&Statornutzahl\\
%		$Q_\textsf{A}$ 		&$\textsf{-}$	&Durchmesserverhältnis\\
%		$q$					&$\textsf{-}$	&Spulen pro Pol und Strang\\
%		$R_\textsf{m}$ 		&$\frac{\textsf{N}}{\textsf{m}^\textsf{2}}$		&Zugfestigkeit\\
%		$R_\textsf{m}$ 		&$\frac{\textsf{A}}{\textsf{V\cdot s}}$	&magnetischer Widerstand\\
%		$R_{\textsf{m,fb,}i}$ &$\frac{\textsf{A}}{\textsf{V\cdot s}}$	&magnetischer Widerstand der $i$-ten Flusssperre\\
%		$R_{\textsf{m,fbe,}i\textsf{,n}}$&$\frac{\textsf{A}}{\textsf{V\cdot s}}$	&magnetischer Ersatzwiderstand für das $i$-te Flusssperrenende\\
%		 & &in negativer Drehrichtung von $\vartheta$\\
%		$R_{\textsf{m,fbe,}i\textsf{,p}}$&$\frac{\textsf{A}}{\textsf{V\cdot s}}$	&magnetischer Ersatzwiderstand für das $i$-te Flusssperrenende\\
%		 & &in positiver Drehrichtung von $\vartheta$\\
%		$R_\textsf{p0.2}$ 	&$\frac{\textsf{N}}{\textsf{m}^\textsf{2}}$		&0.2\,\% Dehngrenze/Streckgrenze\\
%		$R_\textsf{S}$ 		&$\Omega$		&elektrischer Strangwiderstand\\
%		$r$ 				&$\textsf{m}$	&Radius\\
%		$r_\textsf{A}$ 		&$\textsf{m}$	&Radius außen\\
%		$r_\textsf{I}$ 		&$\textsf{m}$	&Radius innen\\
%		$r_\textsf{Si}$ 	&$\textsf{m}$	&Innenradius des Statorblechs\\
%		$r_\textsf{Sh}$ 	&$\textsf{m}$	&Außenradius der Welle im Bereich des Rotorblechpakets\\
%		$r_\textsf{So}$ 	&$\textsf{m}$	&Außenradius des Statorblechs\\
%		$r_\textsf{Ri}$ 	&$\textsf{m}$	&Innenradius des Rotorblechs\\
%		$r_\textsf{Ro}$ 	&$\textsf{m}$	&Außenradius des Rotorblechs\\
%		$\text{r}_{\textsf{fb,}\eta\textsf{,}i}$&$\textsf{-}$&Winkelverhältnis Platzierung der Steuerungspunkte an den\\
%		 & &Flusssperrenenden\\
%		$\text{r}_\textsf{mid}$&$\textsf{-}$&Längenverhältnis Platzierung der Steuerungspunkte im Mittelbereich\\
%		 & &des Flussleitpfads\\
%		$\text{r}_\textsf{q}$&$\textsf{-}$&Längenverhältnis Platzierung der Steuerungspunkte an der $q$-Achse\\
%		$S_\textsf{PA}$ 	&$\textsf{-}$	&Sicherheit gegen vollständige Plastifizierung des Außenteils\\
%		$S_\textsf{R}$ 		&$\textsf{-}$	&Sicherheit gegen Durchrutschen der Welle\\
%		$s$ 				&$\textsf{m}$	&Weg\\
%		$s_\textsf{Q}$ 		&$\textsf{m}$	&Weg einer Feldlinie im Nutöffnungsbereich\\
%		$s_\textsf{Z}$ 		&$\textsf{m}$	&Weg einer Feldlinie im Zahnbereich\\
%		$U$ 				&$\textsf{V}$	&elektrische Spannung\\
%		$U_\textsf{ind}$ 	&$\textsf{V}$	&induzierte Spannung\\
%		$U_\textsf{max}$ 	&$\textsf{m}$	&maximales Übermaß\\
%		$U_\textsf{min}$ 	&$\textsf{m}$	&minimales Übermaß\\
%		$U_\textsf{S}$ 		&$\textsf{V}$	&Strangspannung\\
%		$u$ 				&$\textsf{m}$	&Verschiebung\\
%		$V$					&$\textsf{A}$	&magnetische Spannung/Potential\\
%		$V_\textsf{C}$		&$\textsf{A}$	&magnetische Spannung im Luftspalt für eine Spule\\
%		$V_\textsf{Gr}$		&$\textsf{A}$	&magnetische Spannung im Luftspalt für eine Spulengruppe\\
%		$V_\textsf{R,High}$, $V_\textsf{R,Low}$&$\textsf{A}$	&Randbedingungen für das magn. Potential am Flusssperrenende\\
%		$V_\textsf{Strang}$	&$\textsf{A}$	&magnetische Spannung im Luftspalt für einen Strang\\
%		$V_\delta$			&$\textsf{A}$	&magnetische Spannung im Luftspalt\\
%		$V_{\delta\textsf{,1}}$	&$\textsf{A}$	&magnetische Spannung der Grundwelle\\
%		$W$ 				&$\textsf{-}$	&Spulenweite\\
%		$w_{\textsf{fb,}i}$ 	&$\textsf{m}$	&Breite der $i$-ten Flusssperre\\
%		$w_{\textsf{fbe,}i}$ 	&$\textsf{m}$	&Breite des $i$-ten Flusssperrenendes\\
%		$w_{\textsf{fc,}i}$ 	&$\textsf{m}$	&Breite des $i$-ten Flussleitpfads\\
%		$X$ 				&$\Omega$		&Reaktanz\\
%		$x$ 				&$\textsf{m}$	&(Umfangs-)Koordinate\\
%		$x$, $y$, $z$		&$\textsf{m}$	&statorfeste Koordinaten\\
%		$x$, $\eta$, $\zeta$	&$\textsf{m}$		&rotorfeste Koordinaten\\
%	\end{longtable}
%\end{spacing}
%%
%%
%\newpage
%\section*{Griechische Formelzeichen}
%\begin{spacing}{1.5}
%	\setlength\LTleft{8pt}
%	\setlength\LTright{\fill}
%	\begin{longtable}{@{\extracolsep{\fill}}ccl}
%		\textbf{Symbol}			&\textbf{Einheit}			&\textbf{Beschreibung}                                           \\\midrule\endhead
%		$\alpha_\textsf{fb}$	&$\textsf{rad}$		&Winkel der Flusssperrenbreite an den Enden\\
%		$\alpha_\textsf{k}$		&$\textsf{-}$		&Kerbformzahl\\
%		$\beta$					&$\textsf{rad}$		&Statorstromwinkel (el.)\\
%		$\gamma$				&$\textsf{rad}$		&Umfangswinkel (el.)\\
%		$\gamma_\textsf{m}$		&$\textsf{rad}$				&Rotorpositionswinkel (mech.)\\
%		$\gamma_\textsf{R}$		&$\textsf{rad}$			&rotorfester Umfangswinkel von der $d$-Achse aus (mech.)\\
%		$\gamma_\textsf{S}$		&$\textsf{rad}$			&statorfester Umfangswinkel (mech.)\\
%		$\Delta l_\textsf{q}$ 	&$\textsf{m}$		&Abstand zur $q$-Achse für die Ermittlung der Funktion $f_\textsf{q,fc}(\vartheta)$\\
%		$\Delta M_\textsf{e,rel}$&$\textsf{-}$		&relative Drehmomentwelligkeit\\
%		$\Delta r_\textsf{1}$, $\Delta r_\textsf{2}$&$\textsf{m}$		&Radiusdifferenzen für die Ermittlung der Funktion $f_\textsf{mid,fc}(\vartheta)$\\
%		$\Delta s$ 				&$\textsf{m}$		&Abstand zur Sekante für die Ermittlung der Funktion $f_\textsf{mid,fc}(\vartheta)$\\
%		$\Delta s_\textsf{q}$ 	&$\textsf{m}$		&Abstand zur Normalen auf die $q$-Achse für die Ermittlung der Funktion $f_\textsf{q,fc}(\vartheta)$\\
%		$\Delta V$				&$\textsf{A}$		&Differenz zweier magnetischer Potentiale\\
%		$\Delta \vartheta_{\textsf{fb,}i}$&$\textsf{rad}$&Winkeldifferenz der $i$-ten Flusssperre zwischen Pol $\textsf{A}$ und Pol $\textsf{B}$ \\
%		$\delta$				&$\textsf{m}$		&Luftspaltweite\\
%		$\varepsilon_\textsf{r}$&$\textsf{-}$		&radiale Dehnung\\
%		$\varepsilon_\textsf{t}$&$\textsf{-}$		&tangentiale Dehnung\\
%		$\zeta_\textsf{A}$    	&$\textsf{-}$		&bezogener Plastizitätsdurchmesser\\
%		$\eta_\nu$, $\eta_\xi$	&$\textsf{rad}$		&Ordnungszahlen abhängiger Winkel analytische Drehmomentberechnung\\
%		$\Theta$				&$\textsf{A}$		&Durchflutung\\
%		$\Theta_\textsf{C}$		&$\textsf{A}$		&Spulendurchflutung\\
%		$\Theta_\textsf{Q}$		&$\textsf{A}$		&Nutdurchflutung\\
%		$\Theta_{\textsf{x}\eta}$, $\Theta_{\textsf{x}\zeta}$&$\textsf{kg}\cdot \textsf{m}^\textsf{2}$		&Massendeviationsmomente\\
%		$\vartheta$				&$\textsf{rad}$		&rotorfester Umfangswinkel von der $q$-Achse aus (mech.)\\
%		$\vartheta_{\textsf{fb,}i}$				&$\textsf{rad}$		&Positionswinkel der $i$-ten Flusssperre (mech.)\\
%		$\kappa$				&$\textsf{-}$	&Krümmung\\
%		$\kappa_\textsf{L}$		&$\textsf{-}$	&Verhältnis der Induktivitäten der $d$- und $q$-Achse\\
%		$\lambda$				&$\frac{\textsf{V\cdot s}}{\textsf{A} \cdot \textsf{m}^\textsf{2}}$	&flächenbezogener magnetischer Leitwert\\
%		$\lambda_{\textsf{fb,}j}$ &$\frac{\textsf{V\cdot s}}{\textsf{A}}$		&Leitwert des $j$-ten Teilstücks am Flusssperrenende\\
%		$\lambda_\delta$		&$\frac{\textsf{V\cdot s}}{\textsf{A} \cdot \textsf{m}^\textsf{2}}$		&flächenbezogener magnetischer Leitwert des Luftspalts\\
%		$\lambda_{\delta\textsf{,}j}$&$\frac{\textsf{V\cdot s}}{\textsf{A}}$  &Leitwert des $j$-ten Luftspaltteilstücks\\
%		$\tilde{\lambda}$		&$\textsf{-}$		&dimensionsloser Leitwert des Luftspalts\\
%		$\mu$					&$\textsf{-}$		&relative Permeabilität\\
%		$\mu$					&$\textsf{-}$		&Querkontraktionszahl\\
%		$\mu_\textsf{H}$		&$\textsf{-}$		&Haftreibungskoeffizient\\
%		$\mu_0$					&$\frac{\textsf{V\cdot s}}{\textsf{A\cdot m}}$		&magnetische Feldkonstante\\
%		$\nu$					&$\textsf{-}$		&Ordnungszahl magnetisches Potential/Strombelag\\
%		$\nu_\textsf{S}$		&$\textsf{-}$		&Ordnungszahl Nutöffnungseinfluss\\
%		$\xi$    				&$\textsf{-}$		&Ordnungszahl magnetisches Potential/Strombelag\\
%		$\xi_\textsf{W,max}$    &$\textsf{-}$		&bezogenes Höchstübermaß\\
%		$\rho$					&$\frac{\textsf{kg}}{\textsf{m}^\textsf{3}}$			&Dichte\\
%		$\rho_{i\textsf{,}\nu}$	&$\textsf{-}$			&dimensionslose Geometriefaktoren für analytische Drehmomentberechnung\\
%		$\sigma_\textsf{r}$		&$\frac{N}{\textsf{m}^\textsf{2}}$	&Radialspannung (mech.)\\
%		$\sigma_\textsf{r,A}$		&$\frac{N}{\textsf{m}^\textsf{2}}$	&Radialspannung am Außenradius (mech.)\\
%		$\sigma_\textsf{r,I}$		&$\frac{N}{\textsf{m}^\textsf{2}}$	&Radialspannung am Innenradius (mech.)\\
%		$\sigma_\textsf{t}$		&$\frac{N}{\textsf{m}^\textsf{2}}$	&Tangentialspannung (mech.)\\
%		$\tau_\textsf{p}$		&$\textsf{m}$		&Polteilung\\
%		$\tau_\textsf{Q}$		&$\textsf{m}$		&Nutteilung\\		
%		$\Phi$					&$\textsf{Wb}$		&magnetischer Fluss\\
%		$\Phi_{\textsf{fb,}i}$	&$\textsf{Wb}$		&magnetischer Fluss über die $i$-te Flusssperre\\
%		$\Phi_{\textsf{S,fc,}i}$	&$\textsf{Wb}$		&magnetischer Fluss vom Stator in den $i$-ten Flussleitpfad für $V_\textsf{R}=0$\\
%		$\varphi_\textsf{S}$	&$\textsf{rad}$		&Phasenwinkel zwischen Strangspannung und Strangstrom\\
%		$\chi_{\textsf{fbe,}i\textsf{,n}}$    	&$\textsf{-}$	&Formfunktion für das magnetische Potential über den Enden der Flusssperren\\
%		 & &in negativer Drehrichtung von $\vartheta$\\
%		$\chi_{\textsf{fbe,}i\textsf{,p}}$    	&$\textsf{-}$	&Formfunktion für das magnetische Potential über den Enden der Flusssperren\\
%		 & &in positiver Drehrichtung von $\vartheta$\\
%		$\Psi$					&$\textsf{V\cdot s}$&Flussverkettung\\
%		$\Psi_\textsf{S}$		&$\textsf{V\cdot s}$&Strangflussverkettung\\
%		$\omega_\textsf{m}$		&$\frac{1}{\textsf{s}}$	&Winkelgeschwindigkeit (mech.)\\
%		$\omega_\textsf{S}$		&$\frac{1}{\textsf{s}}$	&Statorkreisfrequenz (el.)\\
%	\end{longtable}
%\end{spacing}
%                    