\section{Industrielle Kommunikation über die OPC UA Schnittstelle}
\label{ch_03Industrielle Kommunikation über die OPC UA Schnittstelle}

Für die Implementierung der Flexibilitätsmaßnahmen auf der MES-Ebene werden die relevanten Datenpunkte über eine OPC UA-Schnittstelle gesteuert. Die OPC Unified Architecture (OPC UA) wurde 2009 von der OPC Foundation veröffentlicht und ist inzwischen in Teilen als internationale Norm IEC 62541 \cite{OPCUnifiedArchitecture2011} standardisiert \cite{ensteOPCUnifiedArchitecture2011}. Diese Architektur findet Anwendung in verschiedenen Bereichen der Fertigungsindustrie, wie beispielsweise in Feldgeräten, Leitsystemen, MES (Manufacturing Execution Systems) und ERP-Systemen (Enterprise Resource Planning) \cite{ensteOPCUnifiedArchitecture2011}. Sie ermöglicht einen konsistenten Datenaustausch zwischen produktionsnahen IT-Systemen und basiert auf einer service-orientierten Architektur für die Kommunikation zwischen Maschinen \cite{ensteOPCUnifiedArchitecture2011}.

Die Infrastruktur von OPC UA umfasst mehrere Modelle, um den Informationsaustausch effizient zu gestalten \cite{OPCUnifiedArchitecture2011}:

\begin{itemize}[label={--}]
	\item \textbf{Informationsmodell:}Ermöglicht die Darstellung von Struktur, Verhalten und Semantik.
	\item \textbf{Nachrichtenmodell:} Ermöglicht die Interaktion zwischen verschiedenen Anwendungen.
	\item \textbf{Kommunikationsmodell:} Unterstützt die Übertragung von Daten zwischen Endpunkten
	\item \textbf{Konformitätsmodell:} Stellt die Interoperabilität zwischen unterschiedlichen Systemen sicher.
\end{itemize}

Die Kommunikationsinfrastruktur von OPC UA unterstützt sowohl das Client-Server-Modell als auch das Publisher-Subscriber-Modell. Letzteres wurde erst 2019 als neue Spezifikation eingeführt \cite{dkedeutschekommissionelektrotechnikelektronikinformationstechnikindinundvdeOPCUnifiedArchitecture2021}. Im Client-Server-Modell initiiert ein Client eine Verbindung und fordert Informationen vom Server an. Ein OPC UA-System kann mehrere Server und Clients enthalten, die gleichzeitig miteinander interagieren können \cite{profanterOPCUAROS2019}. Im Publisher-Subscriber-Modell kann der Server Datenänderungen oder Ereignisse sofort an den Client senden \cite{deveciThoroughAnalysisComparison2022}.\\

Während die meisten Teile der IEC 62541-Serie die OPC UA-Spezifikationen unabhängig von den zugrunde liegenden Implementierungstechnologien definieren, konzentriert sich die IEC 62541-6 \cite{dkedeutschekommissionelektrotechnikelektronikinformationstechnikindinundvdeOPCUnifiedArchitecture2021a} darauf, die spezifischen Netzwerkprotokolle zu bestimmen, die für die Umsetzung der OPC UA-Spezifikationen genutzt werden. Ein Hauptziel von OPC UA ist es, eine größtmögliche Flexibilität für zukünftige technologische Entwicklungen sicherzustellen \cite{ensteOPCUnifiedArchitecture2011}. Neue Kommunikationsprotokolle, die zukünftig entwickelt werden, könnten in die OPC UA-Architektur integriert werden, um für die Datenkommunikation genutzt zu werden. OPC UA bietet das Potenzial, die Kommunikationsinfrastruktur für eine Vielzahl unterschiedlicher Informationsmodelle und Anwendungsbereiche bereitzustellen \cite{fuhrlander-volkerAutomationArchitectureDemand}.\\

Für die Implementierung von OPC UA Servern in verschiedenen Programmiersprachen stehen zahlreiche Software Development Kits (SDKs) zur Verfügung \cite{shilengeOptimizationOperationalInformation2022a}. Diese Server ermöglichen die Bereitstellung von Informationen und Methoden zur Überwachung und Steuerung von Maschinen und Systemen. OPC UA Clients, die sich mit diesen Servern verbinden, haben je nach Berechtigung die Möglichkeit, auf die verschiedenen Knoten innerhalb der vom Server zur Verfügung gestellten Informationsmodelle zuzugreifen und diese zu manipulieren. Die Erstellung von OPC UA Informationsmodellen kann auf verschiedene Weisen erfolgen. Es existieren spezialisierte Modellierungswerkzeuge, die den Prozess der Informationsmodellierung unterstützen [Siem23; UnAu23a]. Alternativ können Informationsmodelle auch manuell mit einem einfachen Texteditor erstellt werden, da sie in Extensible Markup Language (XML) geschrieben werden. XML ist ein leicht lesbares Format, das sowohl für Menschen als auch für Maschinen verständlich ist und somit eine flexible und benutzerfreundliche Gestaltung der Informationsmodelle ermöglicht.\\

OPC UA definiert die Kommunikationsinfrastruktur und das Metamodell zur Beschreibung OPC UA-basierter Informationsmodelle. Das Metamodell umfasst Konstrukte wie Objekte und Variablen, während die Kommunikationsinfrastruktur Implementierungsvorgaben basierend auf Technologien wie TCP oder Web Services beschreibt \cite{dkedeutschekommissionelektrotechnikelektronikinformationstechnikindinundvdeOPCUnifiedArchitecture2021a, ensteOPCUnifiedArchitecture2011}. Diese beiden Komponenten bilden die Grundlage der OPC UA-Spezifikation \cite{ensteOPCUnifiedArchitecture2011}. Die IEC 62541-6 \textcite{dkedeutschekommissionelektrotechnikelektronikinformationstechnikindinundvdeOPCUnifiedArchitecture2021a} spezifiziert unter anderem UA TCP, Hypertext Transfer Protocol Secure (HTTPS) und WebSocket als Transportprotokolle sowie verschiedene Datenkodierungen wie OPC UA Binary, OPC XML und OPC JavaScript Object Notation (JSON) \cite{dkedeutschekommissionelektrotechnikelektronikinformationstechnikindinundvdeOPCUnifiedArchitecture2021a}. Die Unterstützung mehrerer Kommunikationsprotokolle ermöglicht OPC UA eine plattformübergreifende Kommunikation \cite{profanterOPCUAROS2019}.\\

Ein wesentlicher Aspekt von OPC UA ist die integrierte IT-Sicherheit. Diese umfasst die Authentifizierung von Clients und Servern, die Sicherstellung der Datenintegrität und die Verschlüsselung der Kommunikation. OPC UA bietet Mechanismen zur Authentifizierung von Benutzern und Anwendungen (Benutzername und Passwort oder digitale Zertifikate \cite{profanterOPCUAROS2019}) sowie zur Signierung und Verschlüsselung von Nachrichten. Diese Sicherheitsmaßnahmen sind entscheidend, da Automatisierungssysteme zunehmend in Netzwerken betrieben werden, die mit Büro-IT-Systemen oder dem Internet verbunden sind, wodurch sie potenziellen Angriffen ausgesetzt sind \cite{ensteOPCUnifiedArchitecture2011} . Mit der Implementierung von OPC UA Field eXchange (OPC UA FX) für die Echtzeit-Kommunikation auf Feldebene \cite{UAFXPart80}, erstreckt sich die Anwendung von OPC UA über sämtliche Ebenen der Automatisierungshierarchie \cite{shilengeOptimizationOperationalInformation2022a}.\\

Ein bedeutender Vorteil von OPC UA liegt in seinem semantischen Informationsmodell \cite{profanterOPCUAROS2019, deveciThoroughAnalysisComparison2022}. OPC UA zielt darauf ab, die Zusammenarbeit verschiedener Geräte zu ermöglichen, was besonders in Produktionslinien mit vielfältigen Maschinen und Systemen notwendig ist \cite{profanterOPCUAROS2019}. Die Datenmodelle von OPC UA enthalten semantische Informationen, wodurch Clients die Bedeutung der Daten automatisch verstehen können, ohne vorherige Kenntnisse darüber zu haben. Im Gegensatz dazu erfordern andere Protokolle wie MQTT, ROS oder DDS, dass die Clients die Themen der Datenpunkte kennen, um die relevanten Informationen abonnieren zu können \cite{profanterOPCUAROS2019}.\\

Für OPC UA werden standardisierte Datenmodelle, bekannt als OPC UA Companion Specifications, durch Arbeitsgruppen entwickelt. Diese Gruppen bestehen aus Vertretern verschiedener Hersteller und zielen darauf ab, den Datenaustausch zwischen unterschiedlichen Systemen zu standardisieren. Die OPC UA Companion Specification "OPC UA for Machinery" legt grundlegende Informationen für allgemeine Maschinen fest {opc40001-1}. Mit Hilfe von OPC UA-Objekten können Maschinen und deren Komponenten eindeutig identifiziert sowie deren Verhalten überwacht werden {opc40001-1, pp. 21-22}. Dies ermöglicht beispielsweise die Erstellung eines digitalen Typenschilds für Maschinen. Es gibt auch Companion Specifications für spezielle Maschinen, wie beispielsweise "OPC UA for Machine Tools" {opc40501-1} und "OPC UA for Robotics" {opc40010-1}. Bislang existiert jedoch keine Companion Specification für Reinigungsmaschinen.

Companion Specifications (CS) sind von zentraler Bedeutung für OPC UA. Während OPC UA die Rahmenbedingungen für den sicheren Austausch von Informationen festlegt, definieren die CS den Inhalt dieser Informationen. Sie erstellen und dokumentieren Informationsmodelle basierend auf dem OPC UA Metamodell, das Grammatik und Syntax umfasst. Diese Modelle sind entscheidend für den effizienten digitalen Datenaustausch \cite{drathDiskussionspapierInteroperabilitatMit2023}.\\

Durch die Verbindung der standardisierten OPC UA Datenaustauschtechnologie mit der standardisierten Semantik der übertragenen Informationen, wie sie in den Companion Specifications beschrieben wird, entstehen für die Industrie mehrere Vorteile:

- **Reduzierter Entwicklungsaufwand:** Maschinen- und Komponentenhersteller profitieren von einem geringeren Aufwand bei der Entwicklung von Schnittstellen.
- **Erleichterte Systemintegration:** Die Integration von Maschinen, Komponenten und Dienstleistungen beim Betreiber wird vereinfacht.
- **Erhöhte Informationsverfügbarkeit und Transparenz:** Betreiber können Informationen mit höherer Verfügbarkeit und erhöhter Transparenz verarbeiten, ohne aufwendige und fehleranfällige Interpretationen vornehmen zu müssen.

Die OPC UA Companion Specifications bieten spezialisierte Funktionen, die in verschiedenen Bereichen Anwendung finden:

- **Branchenspezifische Informationsmodelle:** Diese Modelle definieren die semantischen Beschreibungen von Daten und Funktionen für unterschiedliche Assets, wie Robotik, Pumpen, Motoren, Auto-ID Geräte, industrielle Küchengeräte und Laboranalysegeräte.
- **Protokollübersetzungen:** Für bestehende Industrieprotokolle wie IO-Link oder ProfiNET wird festgelegt, wie diese eindeutig in OPC UA integriert werden können.
- **Engineering-Übersetzungen:** Die Umsetzung und Abbildung von SPS-Programmiersprachen wie IEC 61131-3 wird unterstützt.

Indem sie klar definierte und standardisierte Informationsmodelle bereitstellen, tragen die OPC UA Companion Specifications wesentlich zur Interoperabilität und Effizienz in der industriellen Kommunikation bei.

Die OPC UA Companion Specification „OPC UA for Energy Management“ legt grundlegende OPC UA-Datenobjekte für das Energiemanagement fest, die von Servern, die diese Spezifikation implementieren, bereitgestellt werden müssen {opc30141}. Diese Datenobjekte sind vielseitig einsetzbar und können von OPC UA-Clients in verschiedenen Szenarien genutzt werden, wie etwa zur Visualisierung und Analyse von Energiedaten oder zur Berechnung des Emissionsfußabdrucks {opc30141, pp. 24-25}. Auch Laststeuerungsmaßnahmen, wie das allgemeine Lastmanagement und die Vermeidung von Lastspitzen, gehören zu den Anwendungsfällen. Die Spezifikation beschränkt sich jedoch auf die Implementierung von Datenobjekten zur Energiemessung und zum Standby-Management. Sie betont, dass diese nützlich für die Umsetzung von Demand-Response-Maßnahmen (DR) sein können, bietet jedoch keine konkrete Umsetzung solcher Maßnahmen wie die Speicherung von Energie oder die Unterbrechung von Prozessen. Abschnitt 6.3 stellt das Datenmodell für diese DR-Maßnahmen detailliert vor (anschauen - aber kommt natürlich nicht in meinen Text)

Für die Implementierung von Demand-Response (DR) ist eine koordinierte Interaktion zwischen Stromnetzbetreibern und Industrieanlagenbetreibern notwendig (fuhrländer) Dies erfordert die Fähigkeit zum Datenaustausch zwischen den Einheiten des Stromnetzes und industriellen Systemen, wie Produktionsmaschinen. Mehrere Forschungsarbeiten, darunter [Claa11; Gil22; Wang22; Zhu22], befassen sich mit der Anwendung von OPC UA in Stromnetzen. Zukünftig könnte OPC UA einen nahtlosen Datentransfer zwischen Stromnetzeinheiten und Produktionsmaschinen ermöglichen, wodurch die Implementierung von DR in der Industrie vereinfacht werden könnte.

Zusammengefasst bietet OPC UA weit mehr als nur die Verbindung zwischen IT und OT. Anstelle eines simplen Kommunikationsprotokolls stellt es eine serviceorientierte Architektur dar, die verschiedene Kommunikationsprotokolle in ihre Struktur einbindet. Eine der wichtigsten Funktionen von OPC UA ist die Möglichkeit, semantische Informationsmodelle zu entwickeln. In der Zukunft könnte diese Technologie die Integration von Fabriken in den Strommarkt erleichtern und dadurch die Umsetzung von Demand-Response (DR) Maßnahmen vereinfachen (Fuhrländer).